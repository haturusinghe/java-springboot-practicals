\documentclass[11pt,a4paper]{article}

\usepackage[utf8]{inputenc}
\usepackage{hyperref}
\usepackage{listings}
\usepackage{color}
\usepackage{graphicx}
\usepackage{enumitem}
\usepackage[margin=1in]{geometry}

% Define colors for syntax highlighting
\definecolor{javablue}{RGB}{0,0,255}
\definecolor{javagreen}{RGB}{0,128,0}
\definecolor{javapurple}{RGB}{128,0,128}

% Configure Java code listings
\lstset{
    language=Java,
    basicstyle=\ttfamily\small,
    keywordstyle=\color{javablue},
    stringstyle=\color{javapurple},
    commentstyle=\color{javagreen},
    numbers=left,
    numberstyle=\tiny,
    numbersep=5pt,
    breaklines=true,
    showspaces=false,
    showstringspaces=false,
    frame=single,
    tabsize=2
}

\title{Java and Object-Oriented Programming Guide}
\author{Comprehensive Guide for Freshman Undergraduates}
\date{\today}

\begin{document}

\maketitle

\tableofcontents

\section{Introduction to Java}

\subsection{History and Overview}
\begin{itemize}
    \item \textbf{Origins:} Java was developed by Sun Microsystems in the mid-1990s (now maintained by Oracle) with the philosophy "Write Once, Run Anywhere" (WORA).
    \item \textbf{Platform Independence:} Java code is compiled to bytecode which runs on the Java Virtual Machine (JVM), making it platform independent.
\end{itemize}

\subsubsection{Core Components}
\begin{itemize}
    \item \textbf{JDK (Java Development Kit):} Contains the compiler (javac), runtime libraries, and tools.
    \item \textbf{JRE (Java Runtime Environment):} The runtime part of Java that runs Java bytecode.
    \item \textbf{JVM (Java Virtual Machine):} The engine that executes bytecode on your device.
\end{itemize}

\subsection{Why Java in Academia?}
\begin{itemize}
    \item \textbf{Object-Oriented:} Emphasizes OOP concepts which are crucial for understanding software design.
    \item \textbf{Strongly Typed \& Robust:} Helps inculcate programming discipline.
    \item \textbf{Wide Industry Use:} A great stepping stone for future career opportunities.
\end{itemize}

% Continue with each section, converting markdown to LaTeX...
% [Content continues with all sections converted to LaTeX format]

\section{Setting Up IntelliJ IDEA Community Edition}

\subsection{Download and Installation}
\begin{enumerate}
    \item \textbf{Download:} Visit the JetBrains website and download the latest Community Edition.
    \item \textbf{Installation:} Follow the installer instructions for your operating system.
    \item \textbf{JDK Setup:} Make sure you have the latest JDK installed.
\end{enumerate}

% Example of a code listing in LaTeX
\begin{lstlisting}[caption=Hello World Example]
public class HelloWorld {
    public static void main(String[] args) {
        System.out.println("Hello, World!");
    }
}
\end{lstlisting}

% [Continue converting all sections...]

\section{Object-Oriented Programming (OOP) Concepts}

\subsection{Classes and Objects}
A class is a blueprint for objects that encapsulates data (attributes) and behavior (methods).

\begin{lstlisting}[caption=Car Class Example]
public class Car {
    // Attributes
    private String model;
    private int year;

    // Constructor
    public Car(String model, int year) {
        this.model = model;
        this.year = year;
    }

    // Method
    public void displayInfo() {
        System.out.println("Model: " + model + ", Year: " + year);
    }
}
\end{lstlisting}

% [Continue with all code examples and explanations...]

\section{Further Learning and Resources}

\subsection{Official Documentation}
\begin{itemize}
    \item \href{https://docs.oracle.com/en/java/}{Oracle's Java Documentation}
    \item \href{https://docs.oracle.com/javase/tutorial/}{Java Tutorials by Oracle}
\end{itemize}

\subsection{Books and Online Courses}
\begin{itemize}
    \item "Head First Java" – a great introductory book
    \item "Effective Java" by Joshua Bloch – for best practices
\end{itemize}

\section{Final Remarks}
This guide has covered everything from setting up IntelliJ IDEA to writing your first Java program and diving deep into OOP principles. When presenting to your students, consider:
\begin{itemize}
    \item \textbf{Interactive Demos:} Run code examples live in IntelliJ
    \item \textbf{Exercises:} Let students code simple examples during the session
    \item \textbf{Q\&A:} Encourage questions to ensure concepts are clear
\end{itemize}

\end{document}
